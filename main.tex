\PassOptionsToPackage{table}{xcolor}
\documentclass[justified,openany,nofonts]{tufte-book}
\usepackage{mooculus}
\usepackage[utf8]{inputenc}
\usepackage{ngerman}
\usepackage{esvect}

%\usepackage{showkeys} %% Useful for debugging
\usepackage{siunitx}
\setcounter{secnumdepth}{2}
\renewcommand{\figurename}{Abbildung}
\usepackage{booktabs,tabularx,multirow,longtable}
\graphicspath{{bilder/}}
\usepackage{arydshln}
\usepackage{scalefnt}
\usepackage{amssymb}
\usepackage{array}
\usepackage{framed}
\usepackage{xcolor}
\usepackage{mdframed}
\usepackage{nicefrac}

% Prints the month name (e.g., January) and the year (e.g., 2008)
\newcommand{\monthyear}{%
  \ifcase\month\or January\or February\or March\or April\or May\or June\or
  July\or August\or September\or October\or November\or
  December\fi\space\number\year
}

% Generates the index
\usepackage{makeidx}
\makeindex

\newcommand{\xrefn}[1]{\ref{#1}}
\renewcommand{\tablename}{Tabelle}


\newenvironment{lemma}{\subsection*{Lemma}}{}
\newenvironment{remark}[1]{\subsection*{Remark: #1}}{}
\newmdenv[hidealllines=true,backgroundcolor=yellow!50]{gelb}

%\def\exam{\null}
\def\pagerdef{\null}

\def\dfont{\bf}
\def\em{\it}           % for emphasis

\newcommand{\ds}{\displaystyle}

\let\ssk\smallskip \let\msk\medskip \let\bsk\bigskip

\usepackage{multicol}
\def\twocol{\begin{multicols}{2}}
\def\endtwocol{\end{multicols}}

\title{Folgen \& Reihen}
\author{Teil I: Folgen}
\publisher{18b 2016}
%\newcommand{\blankpage}{\newpage\hbox{}\thispagestyle{empty}\newpage}

%% % Prints an epigraph and speaker in sans serif, all-caps type.
%% \newcommand{\openepigraph}[2]{%
%%   %\sffamily\fontsize{14}{16}\selectfont
%%   \begin{fullwidth}
%%   \sffamily\large
%%   \begin{doublespace}
%%   \noindent\allcaps{#1}\\% epigraph
%%   \noindent\allcaps{#2}% author
%%   \end{doublespace}
%%   \end{fullwidth}
%% }




\begin{document}
\maketitle

% v.4 copyright page


\begin{fullwidth}
~\vfill
\thispagestyle{empty}
\setlength{\parindent}{0pt}
\setlength{\parskip}{\baselineskip}
Matthias Heimberg

Gymnasium Oberaargau \url{gymo.ch}



\end{fullwidth}


%\tableofcontents


%% \renewcommand{\listtheoremname}{List of Main Theorems}
%% \setcounter{tocdepth}{1}
%% \listoftheorems[numwidth=4em,ignoreall,show={mainTheorem}]



%\chapter*{Die wichtigsten Sätze}% uses ntheorem
%\theoremlisttype{opt}
%\listtheorems{mainTheorem}




\setcounter{chapter}{0}

%%
% Start the main matter (normal chapters)
%\mainmatter
%\tikzexternaldisable

\input 1Folgen



%QUELLEN:
%http://www.physique.lu/lte_physique/classe_10PS-TG/physik_grundlagen.pdf
%https://docs.google.com/file/d/0B4nWIuz6_aEMS0ZmbVI4WHdTLXFFMURob0V1M21Odw/edit

%\bibliography{sample-handout}
%\bibliographystyle{plainnat}

%\finalizeanswers
%\chapter*{Answers to Exercises}
%\small
%\addcontentsline{toc}{chapter}{Answers to Exercises}
%\IfFileExists{answers.tex}{\input{answers}}
%\normalsize
%\backmatter
%\addcontentsline{toc}{chapter}{Index}
\printindex


\end{document}



%sagemathcloud={"zoom_width":95}